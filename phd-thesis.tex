\documentclass[PhD, ngerman, UKenglish, Draft]{scrbook}
%------------------------------------------------------------------------------
% This file contains a skeleton thesis for
% a Physics or Astronomy Institute in the University of Bonn.

% Specify the thesis type as an option: PhD, Master, Diplom, Bachelor.
% Specify the thesis stage as an option: Draft (default), Submit, Final, PILibrary.

% Specify the language(s) in the class and then use babel.
% If you need more than one language, give the default language last,
% e.g. ngerman, UKenglish for a thesis in British (UK) English where you want
% to be able to set the language to German for some part of it.

%------------------------------------------------------------------------------
% Pass TeX Live version to the package.
% Use command pdflatex --version to find out which version you are running.
% twoside=true is suitable for printing, while twoside=false is probably better for PDF version.
\usepackage[twoside=true, texlive=2020]{ubonn-thesis}

%------------------------------------------------------------------------------
% Adjustments to standard biblatex style.
% Change option to backref=false when your thesis is ready to turn off back-referencing.
% Pass the option showurl=false to shorten your bibliography by not including url fields.
\usepackage[backref=true]{ubonn-biblatex}

%------------------------------------------------------------------------------
% Glossary package
\usepackage[acronym,toc,nosuper]{glossaries}

% TikZ packages and libraries
% \usepackage{tikz}
% \usepackage{tikz-3dplot}
% \usepackage{pgfplots}
% \usetikzlibrary{positioning,shapes,arrows}
% \usetikzlibrary{decorations.pathmorphing}
% \usetikzlibrary{decorations.markings}
\usepackage{thesis_defs}

%------------------------------------------------------------------------------
% Instead of colouring  links, cites, table of contents etc.
% put them in a coloured box for the screen version.
% This is probably a good idea when you print your thesis.
% \hypersetup{colorlinks=false,
%   linkbordercolor=blue,citebordercolor=magenta,urlbordercolor=darkgreen
% }

%------------------------------------------------------------------------------
% When writing your thesis it is often helpful to have the date and
% time in the output file. Comment this out for the final version.
\ifoot[\today{} \thistime]{\today{} \thistime}

% In order to check if your labels are referenced try the refcheck package
% \usepackage{refcheck}

%------------------------------------------------------------------------------
% biblatex is included by ubonn-thesis. Look there for the settings used.
% See the options for settings that can be changed easily.
% For further changes copy the \RequirePackage[...]{biblatex} here
% and include ubonn-thesis with the option biblatex=false.

% Specify the bibliography files here and not at the end!
% Use standard_refs-bibtex if you use bibtex or bibtex8
% and standard_refs-biber  if you use biber
\addbibresource{bib/thesis_refs.bib}
\addbibresource{bib/standard_refs-biber.bib}

%------------------------------------------------------------------------------
% The following definitions are used to produce the title pages
% needed at various stages
\newcommand{\thesistitle}{On the application of Machine Learning regression for getting graviational lensing shear estimates}
\newcommand*{\thesisauthor}{Andres Alejandro Navarro Alsina}
\newcommand*{\thesistown}{Aguachica, Kolumbien}
\renewcommand*{\InstituteName}{\AIFA}
\renewcommand*{\inInstitute}{\inAIFA}
\renewcommand*{\InstituteAddress}{\PIaddress}
% Adjust \thesisreferee...text depending on male/female referee
\newcommand*{\thesisrefereeonetext}{1.\ Gutachter}
\newcommand*{\thesisrefereeone}{Prof.\ Dr.\ Peter Schneider}
\newcommand*{\thesisrefereetwotext}{2.\ Gutachterin}
\newcommand*{\thesisrefereetwo}{Prof.\ Dr.\ Cristiano Porciani}
% Date when thesis was submitted (Master/Diplom)
% Year or Month, Year when thesis was submitted (PhD)
\newcommand*{\thesissubmit}{XX.YY.2023}
% \newcommand*{\thesissubmit}{Month 2022}
% Date of thesis examination (PhD)
\newcommand*{\thesispromotion}{XX.YY.2023}
% Month and year of the final printed version of the thesis
\newcommand*{\thesismonth}{MMM}
\newcommand*{\thesisyear}{2023}
\newcommand*{\thesisnumber}{BONN-IR-2023-XXX}
% Dedication
% \newcommand*{\thesisdedication}{}

%------------------------------------------------------------------------------
% The abstract is only needed for the printed version and should be in
% English regardless of the language of the thesis
\newcommand{\thesisabstract}{
  \begin{otherlanguage}{UKenglish}

  \begin{center}
    \textbf{\sffamily\bfseries\upshape\Large Abstract}
  \end{center}

  
  This is your thesis abstract. It may be in a language that is
  different from the rest of your thesis.

\end{otherlanguage}
}

%------------------------------------------------------------------------------
% \includeonly can be used to select which chapters you want to process
% A simple \include command just inserts a \clearpage before and after the file
% Note that \includeonly can be quite picky! Do not forget to put a
% comma after the filename, otherwise it will simply be ignored!
% \includeonly{%
%   thesis_intro,
%   thesis_appendix,
%   thesis_acknowledge
% }

%------------------------------------------------------------------------------
% Give a list of directories where figures can be found. Do not leave
% any spaces in the list and end the directory name with a /
\graphicspath{%
  {figs/}%
  {figs/cover/}%
}

%------------------------------------------------------------------------------
% Make a glossary and a list of acronyms
\makeglossaries

% Glossary entries
\newglossaryentry{cosmicshear}
{
    name=Cosmic shear,
    description={Is the distortion of images of high redshift galaxies through the tidal gravitation force of the large scale matter distribution of universe}
}

\newglossaryentry{stamp}
{
    name=Postage stamp,
    description={Small region of a frame, at the same time a frame is small region of all the ccd image. Postage stamp are not used only for the row image. A postage stamp can make reference to the components of a psf model. The import thing is that it refers to a small region of a frame.}
}

\newglossaryentry{mask}
{
    name=Mask,
    description={Grid used to define an specific region of the sky. In general in image analysis masking referees to spatial filtering. }
}

\newglossaryentry{epoch}
{
    name=Epoch,
    description={Exposure. Might be it is ambiguous because each image measure galaxies at different redshift }
}

\newglossaryentry{gclustering}
{
    name=Galaxy clustering,
    description={Technique in observational cosmology that studies the distribution of galaxies from the angular position in the sky}
}

\newglossaryentry{gclusters}
{
    name=Galaxy clusters,
    description={Technique in observational cosmology that studies cosmology from particular clusters of galaxies }
}


\newglossaryentry{gglensing}
{
    name=Galaxy-galaxy lensing,
    description={Technique that studies the shear of background galaxies, around the center of a foreground galaxy. Very useful to understand the relation between the mass and luminosity of galaxies. And confirm the presence of a dark matter halo}
}

\newglossaryentry{S-N}
{
    name=S/N,
    description={Abrevitation of signal to noise }
}

\newglossaryentry{redmapper}
{
    name=REDMAPPER,
    description={It is a galaxy cluster catalog}
}


\newglossaryentry{balrog}
{
    name=BALROG,
    description={Characterizes selection effects and measurement bias by injecting a realistic ensemble of fake star and galaxy images in the real survey data.}
}

\newglossaryentry{COSEBIs}
{
    name=COSEBIS,
    description={Complete Orthogonal Sets of E/B Integrals. Is a general method for computing E/B modes. }
}

\newglossaryentry{Shape noise}
{
    name=Shape noise,
    description={The intrinsic ellipticity of a galaxy is typically one order of magnitude larger than the weak lensing stretching. Shape noise is the standard deviation of the intrinsic ellipticities of galaxies. Therefore, to overcome shape noise we need a large sample of galaxies.  }
}

\newglossaryentry{Noise bias}
{
    name=Noise bias,
    description={When the image is in low S/N regime, pixel noise on the shapes induce a bias. }
}

\newglossaryentry{Model bias}
{
    name=Model bias,
    description={It appears when the galaxies have complex morphology. }
}

\newglossaryentry{SCAMP}
{
    name=SCAMP,
    description={Reads SExtractor catalogs and computes astrometric and photometric solutions for any arbitrary sequence of FITS images. }
}

\newglossaryentry{SWARP}
{
    name=SWARP,
    description={Is a program that resamples and co-adds together FITS images using any arbitrary astrometric projection defined in the WCS standard. }
}

\newglossaryentry{Astrometric solution}
{
    name=Astrometric solution,
    description={Mapping from pixel(x,y) to
sky coordinates( $\theta$ ,  $\phi$) or WCS). This is another source of errors } 
}

\newacronym{CMB}{cmb}{Cosmic Microwave Background}
\newacronym{cdm}{CDM}{Cold Dark Matter}
\newacronym{WCS}{wcs}{World Coordinate System}
\newacronym{nfw}{NFW}{Navarro-Frenk-White}
\newacronym{sed}{SED}{Spectral Energy Distribution}
\newacronym{psf}{PSF}{Point Spread Function}
\newacronym{ir}{IR}{infrared}
\newacronym{nir}{NIR}{near infrared}
\newacronym{hudf}{HUDF}{Hubble Ultra Deep Field}
\newacronym{sdss}{SDSS}{Sloan Digital Sky Survey}
\newacronym{des}{DES}{Dark Energy Survey}
\newacronym{hst}{\textit{HST}}{\textit{Hubble Space Telescope} }
\newacronym{cosmos}{COSMOS}{Cosmic Evolution Survey}
\newacronym{snr}{SNR}{signal-to-noise ratio}



%\glsaddall

% Draft version - add the word DRAFT on the cover pages
\ifthenelse{\equal{\ThesisVersion}{Draft}}{%
  \usepackage{background}
  \backgroundsetup{contents=DRAFT, color=blue!30}
}

%------------------------------------------------------------------------------
\begin{document}

% Make cover and title pages
\makethesistitle

\pagestyle{scrplain}

%------------------------------------------------------------------------------
% You can add your acknowledgements here - don't forget to also add
% them to \includeonly above
%------------------------------------------------------------------------------
\chapter*{Acknowledgements}
\label{sec:ack}
%------------------------------------------------------------------------------

I would like to thank ...

You should probably use \texttt{\textbackslash chapter*} for
acknowledgements at the beginning of a thesis and
\texttt{\textbackslash chapter} for the end.

%%% Local Variables: 
%%% mode: latex
%%% TeX-master: "../mythesis"
%%% End: 


\thesisabstract
%\input{list_of_publications}


\tableofcontents

\mainmatter
\pagestyle{scrheadings}

% Turn off DRAFT for the following pages
\ifthenelse{\equal{\ThesisVersion}{Draft}}{%
  \backgroundsetup{contents={}}
}{}

%------------------------------------------------------------------------------
% Add your chapters here - don't forget to also add them to \includeonly above
% !TEX root = mythesis.tex

%==============================================================================
\chapter{Introduction}
\label{sec:intro}
%==============================================================================

The introduction usually gives a few pages of introduction to the
whole subject, maybe even starting with the Greeks. \gls{snr}

For more information on \LaTeX{} and the packages that are available
see for example the books of Kopka~\cite{kopka04} and Goossens et
al~\cite{goossens04}.

A lot of useful information on particle physics can be found in the
\enquote{Particle Data Book}~\cite{pdg2010}.

I have resisted the temptation to put a lot of definitions into the
file \texttt{thesis\_defs.sty}, as everyone has their own taste as
to what scheme they want to use for names.
However, a few examples are included to help you get started:
\begin{itemize}
\setlength{\itemsep}{0pt}\setlength{\parskip}{0pt}
\item cross-sections are measured in \si{\pb} and integrated
  luminosity in \si{\invpb};
\item the \KoS is an interesting particle;
\item the missing transverse momentum, \pTmiss, is often called
  missing transverse energy, even though it is calculated using a vector sum.
\end{itemize}
Note that the examples of units assume that you are using the
\textsf{siunitx} package.

It also is probably a good idea to include a few well formatted
references in the thesis skeleton. More detailed suggestions on what
citation types to use can be found in the \enquote{Thesis Guide}~\cite{thesis-guide}:
\begin{itemize}
\item articles in refereed journals~\cite{pdg2010,Aad:2010ey};
\item a book~\cite{Halzen:1984mc};
\item a PhD thesis~\cite{tlodd:2012} and a Diplom thesis~\cite{mergelmeyer:2011};
\item a collection of articles~\cite{lhc:vol1};
\item a conference note~\cite{ATLAS-CONF-2011-008};
\item a preprint~\cite{atlas:perf:2009} (you can also use
  \texttt{@online} or \texttt{@booklet} for such things);
\item something that is only available online~\cite{thesis-guide}.
\end{itemize}

At the end of the introduction it is normal to say briefly what comes
in the following chapters.

The line at the beginning of this file is used by TeXstudio etc.\ to
specify which is the master \LaTeX\ file, so that you can compile your thesis
directly from this file.
If your thesis is called something other than \texttt{mythesis}, adjust it as appropriate.

For demonstration purposes,
we include a figure and a table that a referenced using the \texttt{cleveref} package.
\Cref{fig:nothing} does not show much,
while \cref{tab:little} is not much better.

\begin{figure}[htbp]
  \centering
  \fbox{\textcolor{red}{This is not really a figure!}}
  \caption{A caption for a figure that is not really there.
    Just for fun we can refer to the contents of \cref{tab:little}.}
  \label{fig:nothing}
\end{figure}

\begin{table}[htbp]
  \caption{A table with not much in it.
    Just for fun we can refer to the contents of \cref{fig:nothing}.}
  \label{tab:little}
  \centering
  \begin{tabular}{ll}
    \toprule
    Number & Letter \\
    \midrule
    1 & A \\
    2 & B \\
    \bottomrule  
  \end{tabular}
\end{table}

% Uncomment the following command to get references per chapter.
% Put it inside the file or change \include to \input if you do not want the references
% on a separate page
% \printbibliography[heading=subbibliography]

%------------------------------------------------------------------------------
\appendix
% \part*{Appendix}
% Add your appendices here - don't forget to also add them to \includeonly above
%------------------------------------------------------------------------------
\chapter{Useful information}
\label{sec:app}
%------------------------------------------------------------------------------

In the appendix you usually include extra information that should be
documented in your thesis, but not interrupt the flow.

The \LaTeX\ WikiBook~\cite{latexwiki} is a useful source of information on \LaTeX.

% \printbibliography[heading=subbibliography]

%------------------------------------------------------------------------------
% Use biblatex for the bibliography
% Add bibliography to Table of Contents
% Comment out this command if your references are printed for each chapter.
\printbibliography[heading=bibintoc]

%------------------------------------------------------------------------------
% Include the following lines and comment out \printbibliography if
% you use BiBTeX for the bibliography.
% If you use biblatex package the files should be specified in the preamble.
% \KOMAoptions{toc=bibliography}
% {\raggedright
%   \bibliographystyle{../refs/atlasBibStyleWithTitle.bst}
%   % \bibliographystyle{unsrt}
%   \bibliography{./thesis_refs,../refs/standard_refs-bibtex}
% }

%------------------------------------------------------------------------------
% Declare lists of figures and tables and acknowledgements as backmatter
% Chapter/section numbers are turned off
\backmatter

\listoffigures
\listoftables

%------------------------------------------------------------------------------
% Print the glossary and list of acronyms
\printglossaries

%------------------------------------------------------------------------------
% CV needed when you submit your PhD thesis
%\definecolor{lightgray}{gray}{0.8}
\newcolumntype{L}{>{\raggedleft}p{0.15\textwidth}}
\newcolumntype{R}{p{0.8\textwidth}}
\newcommand\VRule{\color{lightgray}\vrule width 0.5pt}

\thispagestyle{empty}
\section*{Curriculum Vitae}

\subsection*{Personal Details}

\begin{tabular}{L!{\VRule}R}
Name & Johann Schmidt \\
Date of Birth &  \\
Email & abc@physik.uni-def.de \\
Family status & Single
\end{tabular}

\subsection*{Education}

\begin{tabular}{L!{\VRule}R}
1997--2003 & Abitur, ABC Secondary School, Hamburg, Germany\\
2004--2007 & BSc in Physics, Rheinische Friedrich-Wilhelms-Universität, Bonn, Germany.\\
2006 & CERN Summer Student, Geneva, Switzerland. \\
2007--2009 &  MSc in Physics Rheinische Friedrich-Wilhelms-Universität, Bonn, Germany. \\
2009--2012 &  PhD in Physics, Rheinische Friedrich-Wilhelms-Universität, Bonn, Germany. \\
2012 & Advanced Data Analysis School, Frankfurt, Germany.
\end{tabular}

\subsection*{Professional Experience}

\begin{tabular}{L!{\VRule}R}
2004 & Summer Student at CERN, Geneva, Switzerland. \\
2007--2012 & Doctoral work at the University of Bonn, Germany. \\
2008--2009 & Fieldwork at CERN, Geneva, Switzerland.\\
2011 & Talk at the Advanced Physics Conference, Timbucto
\end{tabular}

\subsection*{Languages}
\begin{tabular}{L!{\VRule}R}
German & Mother tongue \\
English & Fluent \\
Russian & Basic
\end{tabular}


\end{document}
